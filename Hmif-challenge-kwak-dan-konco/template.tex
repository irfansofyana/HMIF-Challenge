\documentclass{article}

\usepackage{geometry}
\usepackage{amsmath}
\usepackage{graphicx}
\usepackage{listings}
\usepackage{hyperref}
\usepackage{multicol}
\usepackage{fancyhdr}
\pagestyle{fancy}
\hypersetup{ colorlinks=true, linkcolor=black, filecolor=magenta, urlcolor=cyan}
\geometry{ a4paper, total={170mm,257mm}, top=20mm, right=20mm, bottom=20mm, left=20mm}
\setlength{\parindent}{0pt}
\setlength{\parskip}{1em}
\renewcommand{\headrulewidth}{0pt}
\lhead{Competitive Programming - HMIF Challenge}
\fancyfoot[CE,CO]{\thepage}
\lstset{
    basicstyle=\ttfamily\small,
    columns=fixed,
    extendedchars=true,
    breaklines=true,
    tabsize=2,
    prebreak=\raisebox{0ex}[0ex][0ex]{\ensuremath{\hookleftarrow}},
    frame=none,
    showtabs=false,
    showspaces=false,
    showstringspaces=false,
    prebreak={},
    keywordstyle=\color[rgb]{0.627,0.126,0.941},
    commentstyle=\color[rgb]{0.133,0.545,0.133},
    stringstyle=\color[rgb]{01,0,0},
    captionpos=t,
    escapeinside={(\%}{\%)}
}

\begin{document}

\begin{center}
    \section*{A. Kwak dan Konco} % ganti judul soal

    \begin{tabular}{ | c c | }
        \hline
        Batas Waktu  & 1s \\    % jangan lupa ganti time limit
        Batas Memori & 512MB \\  % jangan lupa ganti memory limit
        \hline
    \end{tabular}
\end{center}

\subsection*{Deskripsi}

% Contoh memformat teks: \textbf{bold}, \textit{italic}, \underline{underline}, $x$.

% Contoh membuat persamaan:

% \[ x_{n+1} = x_{n} - \frac{f(x_{n})}{f'(x_{n})} \]
Kwak telah menyelesaikan pendidikan SMA! Kwak yang sejak SMA sudah hobi dan jago pemrograman ingin melanjutkan hobinya dalam pemrograman tersebut. Oleh karena itu, Kwak memutuskan untuk mengikuti perlombaan ICPC. Perlombaan ini adalah salah satu perlombaan pemrograman yang paling bergengsi untuk mahasiswa.
Berbeda dengan OSN yang pernah diikutinya, ICPC adalah lomba pemrograman untuk tim. Satu buah tim yang mengikuti ICPC terdiri dari 3 orang mahasiswa yang berasal dari Universitas yang sama. Sungguh beruntung nasibnya Kwak. Dua orang teman baiknya yang juga jago pemrograman, yaitu Kwik dan Kwek sekarang menuntut ilmu di kampus yang sama dengan Kwak. Oleh karena itu, akhirnya mereka membentuk tim untuk ICPC!
ICPC adalah lomba pemrograman yang jauh lebih sulit dibandingkan OSN. Karena sebab itulah, mereka bertiga sering sekali melakukan latihan.

Pada suatu hari, mereka bertiga berlatih mengerjakan $N$ soal untuk persiapan ICPC yang semakin dekat. Soal-soal ini diberi nomor dari 1 sampai $N$. Masing-masing dari mereka bertiga membaca soal satu per satu, lalu memberikan tingkat kesulitan relatif menurut pandangan mereka masing-masing untuk setiap soal. Supaya tidak terlalu ribet, mereka sepakat untuk memberikan kesulitan dengan bilangan 1 sampai 5, semakin besar angkanya maka soal tersebut makin susah.
Kwak yang bisa dibilang ketua kelompok, ingin membagi  soal ini menjadi 3 bagian dimana tiap bagian tidak kosong dan merupakan soal-soal yang nomornya berurutan. Setiap soal harus termasuk ke tepat 1 bagian diantara 3 bagian tersebut. Setelah soal-soal tersebut dibagi menjadi 3 bagian, maka setiap orang yang mendapatkan bagiannya akan menghitung tingkat kesulitannya masing-masing dalam latihan,tingkat kesulitan ini senilai dengan penjumlahan kesulitan soal-soal yang akan dikerjakannya. Penjumlahan kesulitan dari Kwak, Kwik, dan Kwek akan menjadi kesulitan tim untuk mengerjakan latihan ini.
Kwak, Kwik, dan Kwek memang ahli dalam pemrograman. Tetapi mereka bertiga juga manusia yang ingin mendapatkan kemudahan. Bantulah Kwak dan temannya untuk menemukan nilai kesulitan tim terkecil tersebut.

% \begin{enumerate}
%     \setlength\itemsep{0pt}
%     \item Contoh penomoran.
%     \item Contoh penomoran.
% \end{enumerate}

% \begin{itemize}
%     \setlength\itemsep{0pt}
%     \item Contoh membuat poin-poin.
%     \item Contoh membuat poin-poin.
% \end{itemize}

% \begin{center}
%     Teks rata tengah
%     % Contoh gambar:
%     % \includegraphics[width=300px]{image-1}
% \end{center}

\subsection*{Format Masukan}

% Baris pertama terdiri dari satu bilangan bulat positif $T$ ($1 \leq T \leq 100.000$), menyatakan banyaknya kasus uji.
% Tiap kasus uji diawali dengan bilangan $N$ ($1 \leq N \leq 100.000$).
% $N$ baris berikutnya terdiri dari 3 bilangan, dengan baris ke-$i$ menyatakan bilangan $A_i$, $B_i$, dan $C_i$.

Baris pertama berisi bilangan bulat $N$, yang menyatakan banyaknya soal yang akan dikerjakan oleh Kwak, Kwik dan Kwek.
Tiga Baris berikutnya masing-masing berisi  buah bilangan, yaitu kesulitan dari soal nomor 1 sampai $N$ menurut Kwak, Kwik, dan Kwek berturut-turut.

\subsection*{Format Keluaran}
Sebuah baris berisi bilangan yang merupakan nilai kesulitan tim terkecil yang mungkin bagi Kwak, Kwik, dan Kwek.
\\

\begin{multicols}{2}
\subsection*{Contoh Masukan 1}
\begin{lstlisting}
3
1 3 3
1 1 1
1 2 3
\end{lstlisting}
\columnbreak
\subsection*{Contoh Keluaran 1}
\begin{lstlisting}
4
\end{lstlisting}
\vfill
\null
\end{multicols}

\begin{multicols}{2}
\subsection*{Contoh Masukan 2}
\begin{lstlisting}
7
3 3 4 1 3 4 4
4 2 5 1 5 5 4
5 5 1 3 4 4 4
\end{lstlisting}
\columnbreak
\subsection*{Contoh Keluaran 2}
\begin{lstlisting}
19
\end{lstlisting}
\vfill
\null
\end{multicols}

% \subsection*{Penjelasan}
% Jika dibutuhkan, tambahkan penjelasan di sini

\pagebreak

\end{document}