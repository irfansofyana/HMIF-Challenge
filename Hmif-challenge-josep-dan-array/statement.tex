\documentclass{article}

\usepackage{geometry}
\usepackage{amsmath}
\usepackage{graphicx}
\usepackage{listings}
\usepackage{hyperref}
\usepackage{multicol}
\usepackage{fancyhdr}
\pagestyle{fancy}
\hypersetup{ colorlinks=true, linkcolor=black, filecolor=magenta, urlcolor=cyan}
\geometry{ a4paper, total={170mm,257mm}, top=20mm, right=20mm, bottom=20mm, left=20mm}
\setlength{\parindent}{0pt}
\setlength{\parskip}{1em}
\renewcommand{\headrulewidth}{0pt}
\lhead{Competitive Programming - HMIF Challenge}
\fancyfoot[CE,CO]{\thepage}
\lstset{
    basicstyle=\ttfamily\small,
    columns=fixed,
    extendedchars=true,
    breaklines=true,
    tabsize=2,
    prebreak=\raisebox{0ex}[0ex][0ex]{\ensuremath{\hookleftarrow}},
    frame=none,
    showtabs=false,
    showspaces=false,
    showstringspaces=false,
    prebreak={},
    keywordstyle=\color[rgb]{0.627,0.126,0.941},
    commentstyle=\color[rgb]{0.133,0.545,0.133},
    stringstyle=\color[rgb]{01,0,0},
    captionpos=t,
    escapeinside={(\%}{\%)}
}

\begin{document}

\begin{center}
    \section*{Josep dan array} % ganti judul soal

    \begin{tabular}{ | c c | }
        \hline
        Batas Waktu  & 1s \\    % jangan lupa ganti time limit
        Batas Memori & 64MB \\  % jangan lupa ganti memory limit
        \hline
    \end{tabular}
\end{center}

\subsection*{Deskripsi}

% Contoh memformat teks: \textbf{bold}, \textit{italic}, \underline{underline}, $x$.

% Contoh membuat persamaan:

% \[ x_{n+1} = x_{n} - \frac{f(x_{n})}{f'(x_{n})} \]
Josep memiliki array yang berisi $N$ buah bilangan bulat. Josep yang gabut akan bermain main 
dengan array tersebut. Permainan akan berlangsung selama $N$ putaran. Pada setiap putarannya, 
Josep akan melakukan hal berikut:
\begin{itemize}
    \item Josep akan memilih sebuah elemen pada array dan akan menghapusnya. Setiap kali menghapus
    suatu elemen, maka Josep akan mendapatkan poin sebesar $min(a, b)$ dimana $a$ dan $b$ adalah 
    elemen yang bersebelahan langsung dengan yang dihapus. Namun, apabila elemen yang dihapus adalah elemen pertama atau 
    terakhir, maka Josep akan mendapatkan 0 poin
    
    \item Setelah elemen yang dipilih Josep itu dihapus, maka Josep akan menggabungkan bagian array sebelah kiri 
    elemen yang dihapus dengan bagian array sebelah kanan elemen yang dihapus
\end{itemize}

Josep adalah orang yang senang memberikan tantangan. Ia memberikan tantangan kepada Anda sebagai peserta HMIF
Challenge untuk mencari berapakah maksimal poin yang bisa didapatkan oleh dirinya?

% \begin{enumerate}
%     \setlength\itemsep{0pt}
%     \item Contoh penomoran.
%     \item Contoh penomoran.
% \end{enumerate}

% \begin{itemize}
%     \setlength\itemsep{0pt}
%     \item Contoh membuat poin-poin.
%     \item Contoh membuat poin-poin.
% \end{itemize}

% \begin{center}
%     Teks rata tengah
%     % Contoh gambar:
%     % \includegraphics[width=300px]{image-1}
% \end{center}

\subsection*{Format Masukan}

% Baris pertama terdiri dari satu bilangan bulat positif $T$ ($1 \leq T \leq 100.000$), menyatakan banyaknya kasus uji.
% Tiap kasus uji diawali dengan bilangan $N$ ($1 \leq N \leq 100.000$).
% $N$ baris berikutnya terdiri dari 3 bilangan, dengan baris ke-$i$ menyatakan bilangan $A_i$, $B_i$, dan $C_i$.

Baris pertama berisi bilangan bulat $N$ (1 $\leq N \leq 500000$), yang menyatakan banyaknya bilangan pada array.
Baris berikutnya terdiri dari $N$ buah bilangan bulat $A_i$ (1 $\leq A_i \leq 1000000$) yang merupakan elemen array yang dimiliki oleh Josep.

\subsection*{Format Keluaran}
Jumlah poin maksimal yang bisa didapatkan oleh Josep
\\

\begin{multicols}{2}
\subsection*{Contoh Masukan 1}
\begin{lstlisting}
5
3 1 5 2 6
\end{lstlisting}
\columnbreak
\subsection*{Contoh Keluaran 1}
\begin{lstlisting}
11
\end{lstlisting}
\vfill
% \null
\end{multicols}

\begin{multicols}{2}
\subsection*{Contoh Masukan 2}
\begin{lstlisting}
5
1 100 101 100 1
\end{lstlisting}
\columnbreak
\subsection*{Contoh Keluaran 2}
\begin{lstlisting}
102
\end{lstlisting}
\vfill
% \null
\end{multicols}



% \subsection*{Penjelasan}
% Jika dibutuhkan, tambahkan penjelasan di sini

\pagebreak

\end{document}