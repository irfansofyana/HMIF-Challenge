\documentclass{article}

\usepackage{geometry}
\usepackage{amsmath}
\usepackage{graphicx}
\usepackage{listings}
\usepackage{hyperref}
\usepackage{multicol}
\usepackage{fancyhdr}
\usepackage{physics}
\pagestyle{fancy}
\hypersetup{ colorlinks=true, linkcolor=black, filecolor=magenta, urlcolor=cyan}
\geometry{ a4paper, total={170mm,257mm}, top=20mm, right=20mm, bottom=20mm, left=20mm}
\setlength{\parindent}{0pt}
\setlength{\parskip}{1em}
\renewcommand{\headrulewidth}{0pt}
\lhead{Competitive Programming - HMIF Challenge}
\fancyfoot[CE,CO]{\thepage}
\lstset{
    basicstyle=\ttfamily\small,
    columns=fixed,
    extendedchars=true,
    breaklines=true,
    tabsize=2,
    prebreak=\raisebox{0ex}[0ex][0ex]{\ensuremath{\hookleftarrow}},
    frame=none,
    showtabs=false,
    showspaces=false,
    showstringspaces=false,
    prebreak={},
    keywordstyle=\color[rgb]{0.627,0.126,0.941},
    commentstyle=\color[rgb]{0.133,0.545,0.133},
    stringstyle=\color[rgb]{01,0,0},
    captionpos=t,
    escapeinside={(\%}{\%)}
}

\begin{document}

\begin{center}
    \section*{Permutasi} % ganti judul soal

    \begin{tabular}{ | c c | }
        \hline
        Batas Waktu  & 1s \\    % jangan lupa ganti time limit
        Batas Memori & 64MB \\  % jangan lupa ganti memory limit
        \hline
    \end{tabular}
\end{center}

\subsection*{Deskripsi}

% Contoh memformat teks: \textbf{bold}, \textit{italic}, \underline{underline}, $x$.

% Contoh membuat persamaan:

% \[ x_{n+1} = x_{n} - \frac{f(x_{n})}{f'(x_{n})} \]
Permutasi adalah istilah yang sudah sering Anda dengar bukan? Sekarang pun Anda akan menghadapi persoalan mengenai permutasi.
Definisikan $P$ sebagai permutasi $N$ bilangan asli pertama [1, N]. Lalu kita juga mendefinisikan $pos[i]$ sebagai posisi bilangan i
pada permutasi $P$ dengan 1-\textit{indexing}.

$P$ disebut sebagai permutasi yang mutlak apabila untuk setiap i $\in$ [1, N] berlaku \(\lvert {pos[i]-i}\rvert\) = $K$. Nah, sekarang Anda diberikan nilai $N$ dan 
$K$. Tugas Anda adalah mencari permutasi mutlak leksikografi terkecil $P$. Jika tidak mungkin didapatkan permutasi mutlak, cetaklah -1.

\textbf{Keterangan}:
Urutan leksikografi pada dua buah permutasi $U$ dan $V$ dengan banyak elemen yang sama adalah sebagai berikut: permutasi $U$ dikatakan lebih kecil secara
leksikografi daripada permutasi $V$ apabila pada indeks terkecil $i$ yang mana $U[i]$ berbeda 
dengan $V[i]$ maka berlaku $U[i]$ $<$ $V[i]$

Contoh: 
[1, 2, 3] lebih kecil dari [1, 3, 2]

% \begin{enumerate}
%     \setlength\itemsep{0pt}
%     \item Contoh penomoran.
%     \item Contoh penomoran.
% \end{enumerate}

% \begin{itemize}
%     \setlength\itemsep{0pt}
%     \item Contoh membuat poin-poin.
%     \item Contoh membuat poin-poin.
% \end{itemize}

% \begin{center}
%     Teks rata tengah
%     % Contoh gambar:
%     % \includegraphics[width=300px]{image-1}
% \end{center}

\subsection*{Format Masukan}

% Baris pertama terdiri dari satu bilangan bulat positif $T$ ($1 \leq T \leq 100.000$), menyatakan banyaknya kasus uji.
% Tiap kasus uji diawali dengan bilangan $N$ ($1 \leq N \leq 100.000$).
% $N$ baris berikutnya terdiri dari 3 bilangan, dengan baris ke-$i$ menyatakan bilangan $A_i$, $B_i$, dan $C_i$.

Format masukan terdiri dari satu baris, yaitu berisi bilangan bulat $N$ ($1 \leq N \leq 100000$) dan 
$K$ ($0 \leq k < N $).

\subsection*{Format Keluaran}
Sebuah permutasi $P$ terkecil yang mungkin sehingga ia permutasi mutlak, atau -1 jika tidak ada yang merupakan permutasi mutlak.

\begin{multicols}{2}
\subsection*{Contoh Masukan 1}
\begin{lstlisting}
3 0
\end{lstlisting}
\columnbreak
\subsection*{Contoh Keluaran 1}
\begin{lstlisting}
1 2 3
\end{lstlisting}
\end{multicols}


% \subsection*{Penjelasan}
% Jika dibutuhkan, tambahkan penjelasan di sini

\pagebreak

\end{document}