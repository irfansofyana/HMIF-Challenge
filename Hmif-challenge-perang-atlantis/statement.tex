\documentclass{article}

\usepackage{geometry}
\usepackage{amsmath}
\usepackage{graphicx}
\usepackage{listings}
\usepackage{hyperref}
\usepackage{multicol}
\usepackage{fancyhdr}
\usepackage{physics}
\pagestyle{fancy}
\hypersetup{ colorlinks=true, linkcolor=black, filecolor=magenta, urlcolor=cyan}
\geometry{ a4paper, total={170mm,257mm}, top=20mm, right=20mm, bottom=20mm, left=20mm}
\setlength{\parindent}{0pt}
\setlength{\parskip}{1em}
\renewcommand{\headrulewidth}{0pt}
\lhead{Competitive Programming - HMIF Challenge}
\fancyfoot[CE,CO]{\thepage}
\lstset{
    basicstyle=\ttfamily\small,
    columns=fixed,
    extendedchars=true,
    breaklines=true,
    tabsize=2,
    prebreak=\raisebox{0ex}[0ex][0ex]{\ensuremath{\hookleftarrow}},
    frame=none,
    showtabs=false,
    showspaces=false,
    showstringspaces=false,
    prebreak={},
    keywordstyle=\color[rgb]{0.627,0.126,0.941},
    commentstyle=\color[rgb]{0.133,0.545,0.133},
    stringstyle=\color[rgb]{01,0,0},
    captionpos=t,
    escapeinside={(\%}{\%)}
}

\begin{document}

\begin{center}
    \section*{Perang Atlantis} % ganti judul soal

    \begin{tabular}{ | c c | }
        \hline
        Batas Waktu  & 2s \\    % jangan lupa ganti time limit
        Batas Memori & 64MB \\  % jangan lupa ganti memory limit
        \hline
    \end{tabular}
\end{center}

\subsection*{Deskripsi}

% Contoh memformat teks: \textbf{bold}, \textit{italic}, \underline{underline}, $x$.

% Contoh membuat persamaan:

% \[ x_{n+1} = x_{n} - \frac{f(x_{n})}{f'(x_{n})} \]
Setelah Atlantis ditemukan, seluruh negara di dunia berebutan menguasai Atlantis. Indonesia tidak mau kalah, pemerintah menunjuk Pak Josep untuk memimpin perebutan kekuasaan. 
Setelah dipelajari, ternyata Atlantis terdiri dari $N$ buah kota, dengan $M$ jalan yang tiap jalannya menghubungkan 2 kota. 
Dijamin terdapat jalan yang menghubungkan satu kota dengan kota lainnya.

Terdapat $C$ negara yang sedang berperang di Atlantis. Seluruh negara berusaha merebut kota sebanyak-banyaknya, karena tiap kota memiliki nilai kekayaan yang berbeda-beda. Kota ke-i memiliki kekayaan senilai $V_i$ dan dikuasai oleh $W_i$. Beberapa kota yang terhubung langsung dan dimiliki suatu negara 
disebut sebagai area kekuasaan. Artinya, kota $A$ dan $B$ berada pada area kekuasaan yang sama jika dan hanya jika ada rangkaian jalan dari $A$ ke $B$ tanpa melewati kota milik negara lain.

Untuk memikirkan strategi terbaik, Pak Josep butuh program yang memudahkan dia mencari kekayaan kota maksimum pada suatu area kekuasaan. Namun, bisa saja nilai kekayaan berubah-ubah (sudah diambil atau ditemukan kekayaan tersembunyi). Mengingat tugas ini tidak mudah, Pak Josep meminta bantuan kalian, sebagai peserta HMIF Challenge 2019!


% \begin{enumerate}
%     \setlength\itemsep{0pt}
%     \item Contoh penomoran.
%     \item Contoh penomoran.
% \end{enumerate}

% \begin{itemize}
%     \setlength\itemsep{0pt}
%     \item Contoh membuat poin-poin.
%     \item Contoh membuat poin-poin.
% \end{itemize}

% \begin{center}
%     Teks rata tengah
%     % Contoh gambar:
%     % \includegraphics[width=300px]{image-1}
% \end{center}

\subsection*{Format Masukan}

% Baris pertama terdiri dari satu bilangan bulat positif $T$ ($1 \leq T \leq 100.000$), menyatakan banyaknya kasus uji.
% Tiap kasus uji diawali dengan bilangan $N$ ($1 \leq N \leq 100.000$).
% $N$ baris berikutnya terdiri dari 3 bilangan, dengan baris ke-$i$ menyatakan bilangan $A_i$, $B_i$, dan $C_i$.

Format masukan terdiri dari satu baris, yaitu berisi bilangan bulat $N$ ($1 \leq N \leq 100000$) dan 
$K$ ($0 \leq k < N $).

\subsection*{Format Keluaran}
Sebuah permutasi $P$ terkecil yang mungkin sehingga ia permutasi mutlak, atau -1 jika tidak ada yang merupakan permutasi mutlak.

\begin{multicols}{2}
\subsection*{Contoh Masukan 1}
\begin{lstlisting}
3 0
\end{lstlisting}
\columnbreak
\subsection*{Contoh Keluaran 1}
\begin{lstlisting}
1 2 3
\end{lstlisting}
\end{multicols}


% \subsection*{Penjelasan}
% Jika dibutuhkan, tambahkan penjelasan di sini

\pagebreak

\end{document}