\documentclass{article}

\usepackage{geometry}
\usepackage{amsmath}
\usepackage{graphicx}
\usepackage{listings}
\usepackage{hyperref}
\usepackage{multicol}
\usepackage{fancyhdr}
\pagestyle{fancy}
\hypersetup{ colorlinks=true, linkcolor=black, filecolor=magenta, urlcolor=cyan}
\geometry{ a4paper, total={170mm,257mm}, top=20mm, right=20mm, bottom=20mm, left=20mm}
\setlength{\parindent}{0pt}
\setlength{\parskip}{1em}
\renewcommand{\headrulewidth}{0pt}
\lhead{Competitive Programming - HMIF Challenge}
\fancyfoot[CE,CO]{\thepage}
\lstset{
    basicstyle=\ttfamily\small,
    columns=fixed,
    extendedchars=true,
    breaklines=true,
    tabsize=2,
    prebreak=\raisebox{0ex}[0ex][0ex]{\ensuremath{\hookleftarrow}},
    frame=none,
    showtabs=false,
    showspaces=false,
    showstringspaces=false,
    prebreak={},
    keywordstyle=\color[rgb]{0.627,0.126,0.941},
    commentstyle=\color[rgb]{0.133,0.545,0.133},
    stringstyle=\color[rgb]{01,0,0},
    captionpos=t,
    escapeinside={(\%}{\%)}
}

\begin{document}

\begin{center}
    \section*{Jumlahan Parsial} % ganti judul soal

    \begin{tabular}{ | c c | }
        \hline
        Batas Waktu  & 1s \\    % jangan lupa ganti time limit
        Batas Memori & 256MB \\  % jangan lupa ganti memory limit
        \hline
    \end{tabular}
\end{center}

\subsection*{Deskripsi}

% Contoh memformat teks: \textbf{bold}, \textit{italic}, \underline{underline}, $x$.

% Contoh membuat persamaan:

% \[ x_{n+1} = x_{n} - \frac{f(x_{n})}{f'(x_{n})} \]

Josep memiliki sebuah array $A$ yang terdiri dari $N$ bilangan bulat. Setiap elemen array tersebut diberikan 
indeks dari 1 sampai dengan $N$. Lalu Josep mendefinisikan 1 langkah operasi berikut ini ke dalam 2 langkah
\begin{enumerate}
    \item Pertama, Josep akan membuat sebuah array parsial sum (diberi nama $S$) yang berisi $N$ elemen dimana
          ${S_i} = (\sum_{j=1}^ia_j) $ mod $(10^9 + 7)$
    \item Kedua, array $A$ yang Josep miliki akan digantikan dengan nilainya dengan array $S$. Dengan kata lain
    Josep akan menjalankan operasi $A_i = S_i$ untuk (1 $\leq i \leq N $)
\end{enumerate}

Josep adalah orang yang memiliki \textit{curiousity} yang tinggi. Oleh sebab itu, ia pun penasaran apabila
ia menjalankan operasi tersebut sebanyak $K$ kali pada array $A$, akan menjadi seperti apa array $A$ pada akhirnya?


% \begin{enumerate}
%     \setlength\itemsep{0pt}
%     \item Contoh penomoran.
%     \item Contoh penomoran.
% \end{enumerate}

% \begin{itemize}
%     \setlength\itemsep{0pt}
%     \item Contoh membuat poin-poin.
%     \item Contoh membuat poin-poin.
% \end{itemize}

% \begin{center}
%     Teks rata tengah
%     % Contoh gambar:
%     % \includegraphics[width=300px]{image-1}
% \end{center}

\subsection*{Format Masukan}

% Baris pertama terdiri dari satu bilangan bulat positif $T$ ($1 \leq T \leq 100.000$), menyatakan banyaknya kasus uji.
% Tiap kasus uji diawali dengan bilangan $N$ ($1 \leq N \leq 100.000$).
% $N$ baris berikutnya terdiri dari 3 bilangan, dengan baris ke-$i$ menyatakan bilangan $A_i$, $B_i$, dan $C_i$.

Baris pertama terdiri dari dua bilangan $N$ dan $K$ (1 $\leq N \leq 2000, 0 \leq K \leq 10^9$).
Baris berikutnya berisi array $A$ yang terdiri dari $N$ buah bilangan bulat $A_i$ (1 $ \leq A_i \leq 10^9 $)

\subsection*{Format Keluaran}
Satu baris berisi $N$ buah bilangan, yaitu array $A$ setelah dilakukan $K$ buah operasi sesuai deskripsi soal.
\begin{multicols}{2}
\subsection*{Contoh Masukan 1}
\begin{lstlisting}
3 1
1 2 3
\end{lstlisting}
% \vfill
\null
\columnbreak
\subsection*{Contoh Keluaran 1}
\begin{lstlisting}
1 3 6
\end{lstlisting}
\end{multicols}


% \subsection*{Penjelasan}
% Jika dibutuhkan, tambahkan penjelasan di sini

\pagebreak

\end{document}