\documentclass{article}

\usepackage{geometry}
\usepackage{amsmath}
\usepackage{graphicx}
\usepackage{listings}
\usepackage{hyperref}
\usepackage{multicol}
\usepackage{fancyhdr}
\pagestyle{fancy}
\hypersetup{ colorlinks=true, linkcolor=black, filecolor=magenta, urlcolor=cyan}
\geometry{ a4paper, total={170mm,257mm}, top=20mm, right=20mm, bottom=20mm, left=20mm}
\setlength{\parindent}{0pt}
\setlength{\parskip}{1em}
\renewcommand{\headrulewidth}{0pt}
\lhead{Competitive Programming - HMIF Challenge}
\fancyfoot[CE,CO]{\thepage}
\lstset{
    basicstyle=\ttfamily\small,
    columns=fixed,
    extendedchars=true,
    breaklines=true,
    tabsize=2,
    prebreak=\raisebox{0ex}[0ex][0ex]{\ensuremath{\hookleftarrow}},
    frame=none,
    showtabs=false,
    showspaces=false,
    showstringspaces=false,
    prebreak={},
    keywordstyle=\color[rgb]{0.627,0.126,0.941},
    commentstyle=\color[rgb]{0.133,0.545,0.133},
    stringstyle=\color[rgb]{01,0,0},
    captionpos=t,
    escapeinside={(\%}{\%)}
}

\begin{document}

\begin{center}
    \section*{A. Kwak dan Konco} % ganti judul soal

    \begin{tabular}{ | c c | }
        \hline
        Batas Waktu  & 2s \\    % jangan lupa ganti time limit
        Batas Memori & 64MB \\  % jangan lupa ganti memory limit
        \hline
    \end{tabular}
\end{center}

\subsection*{Deskripsi}

% Contoh memformat teks: \textbf{bold}, \textit{italic}, \underline{underline}, $x$.

% Contoh membuat persamaan:

% \[ x_{n+1} = x_{n} - \frac{f(x_{n})}{f'(x_{n})} \]
Suatu Provinsi terdiri dari $N$ kota yang dinomori 1 hingga $N$. Untuk menghemat pembangunan jalan tol, pemerintah hanya membangun jalan-jalan utama. Jika suatu kota sudah terhubung dengan jalan utama, pemerintah tidak akan lagi membangun jalan pada kota tersebut. Sehingga, hanya ada ${N-1}$ jalan yang menghubungkan seluruh kota.

Setiap jalan yang menghubungkan kota $U_i$ dan $V_i$ memiliki tarif sebesar $W_i$. Namun, pemerintah memberikan subsidi kepada pengguna tol sehingga mereka hanya perlu membayar tarif termahal dari setiap jalan yang dilaluinya.

Atas dasar penasaran dan iseng, gubernur provinsi memberikan kuis kepada rakyatnya. Terdapat $Q$ pertanyaan yang terdiri dari 2 bilangan bulat, $L$ dan $R$ ( $L$ $\leq$ $R$ ). Untuk setiap pertanyaan, gubernur meminta anda untuk menghitung banyaknya rute berbeda yang mengharuskan pengguna jalan membayar jalan tol dalam rentang $[L,R]$ (inklusif).

Note: rute dari kota A ke kota B sama dengan rute dari kota B ke kota A. Dengan kata lain, $\{A, B\}$ sama dengan $\{B, A\}$ .

% \begin{enumerate}
%     \setlength\itemsep{0pt}
%     \item Contoh penomoran.
%     \item Contoh penomoran.
% \end{enumerate}

% \begin{itemize}
%     \setlength\itemsep{0pt}
%     \item Contoh membuat poin-poin.
%     \item Contoh membuat poin-poin.
% \end{itemize}

% \begin{center}
%     Teks rata tengah
%     % Contoh gambar:
%     % \includegraphics[width=300px]{image-1}
% \end{center}

\subsection*{Format Masukan}

% Baris pertama terdiri dari satu bilangan bulat positif $T$ ($1 \leq T \leq 100.000$), menyatakan banyaknya kasus uji.
% Tiap kasus uji diawali dengan bilangan $N$ ($1 \leq N \leq 100.000$).
% $N$ baris berikutnya terdiri dari 3 bilangan, dengan baris ke-$i$ menyatakan bilangan $A_i$, $B_i$, dan $C_i$.

Baris pertama terdiri dari dua bilangan $N$ dan $Q$ (1 $\leq N, Q \leq 100000$).
${N-1}$ baris berikutnya masing-masing terdiri dari 3 bilangan $U_i$, $V_i$ (1 $\leq U_i, V_i \leq 100000$) dan $W_i$ (1 $\leq W_i \leq 1000000000$) yang menandakan bahwa terdapat jalan yang 
menghubungkan kota $U_i$ dan $V_i$ dengan tarif sebesar $W_i$.
$Q$ baris selanjutnya masing-masing terdiri dari 2 bilangan bilangan $L_i$ dan $R_i$ (1 $\leq L_i \leq R_i \leq 1000000000$)

\subsection*{Format Keluaran}
Sebuah baris berisi bilangan $MAX$ terkecil yang mungkin.
\begin{multicols}{2}
\subsection*{Contoh Masukan 1}
\begin{lstlisting}
5 5
1 2 3
1 4 2
2 5 6
3 4 1
1 1
1 2
2 3
2 5
1 6
\end{lstlisting}
\vfill
\null
\columnbreak
\subsection*{Contoh Keluaran 1}
\begin{lstlisting}
1
3
5
5
10
\end{lstlisting}
\end{multicols}


% \subsection*{Penjelasan}
% Jika dibutuhkan, tambahkan penjelasan di sini

\pagebreak

\end{document}