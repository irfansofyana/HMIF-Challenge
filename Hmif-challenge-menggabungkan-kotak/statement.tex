\documentclass{article}

\usepackage{geometry}
\usepackage{amsmath}
\usepackage{graphicx}
\usepackage{listings}
\usepackage{hyperref}
\usepackage{multicol}
\usepackage{fancyhdr}
\pagestyle{fancy}
\hypersetup{ colorlinks=true, linkcolor=black, filecolor=magenta, urlcolor=cyan}
\geometry{ a4paper, total={170mm,257mm}, top=20mm, right=20mm, bottom=20mm, left=20mm}
\setlength{\parindent}{0pt}
\setlength{\parskip}{1em}
\renewcommand{\headrulewidth}{0pt}
\lhead{Competitive Programming - HMIF Challenge}
\fancyfoot[CE,CO]{\thepage}
\lstset{
    basicstyle=\ttfamily\small,
    columns=fixed,
    extendedchars=true,
    breaklines=true,
    tabsize=2,
    prebreak=\raisebox{0ex}[0ex][0ex]{\ensuremath{\hookleftarrow}},
    frame=none,
    showtabs=false,
    showspaces=false,
    showstringspaces=false,
    prebreak={},
    keywordstyle=\color[rgb]{0.627,0.126,0.941},
    commentstyle=\color[rgb]{0.133,0.545,0.133},
    stringstyle=\color[rgb]{01,0,0},
    captionpos=t,
    escapeinside={(\%}{\%)}
}

\begin{document}

\begin{center}
    \section*{Menggabungkan Kotak} % ganti judul soal

    \begin{tabular}{ | c c | }
        \hline
        Batas Waktu  & 1s \\    % jangan lupa ganti time limit
        Batas Memori & 64MB \\  % jangan lupa ganti memory limit
        \hline
    \end{tabular}
\end{center}

\subsection*{Deskripsi}

% Contoh memformat teks: \textbf{bold}, \textit{italic}, \underline{underline}, $x$.

% Contoh membuat persamaan:

% \[ x_{n+1} = x_{n} - \frac{f(x_{n})}{f'(x_{n})} \]
Josep memiliki $N$ kotak yang disusun berjejer, dinomori dari 1 hingga $N$ dari kiri ke kanan. Karena Josep sangat menyukai permen, maka dia membeli sejumlah permen dan memasukkannya ke dalam kotak-kotak tersebut. Tepatnya, Josep memasukkan $P_i$ permen pada kotak ke-i.

Josep kemudian ingat bahwa Michella besok berulangtahun. Karena Michella juga menyukai permen, maka Josep ingin memberikan permennya. Josep hanya ingin memberikan maksimal satu kotak kepada Michella. Supaya jumlah permen yang diberikan tidak sedikit, maka ia akan menggabung beberapa kotak menjadi satu kotak.

Karena Josep sangat menyukai tantangan, maka dia membuat beberapa syarat dalam menggabung kotak-kotak tersebut:

\begin{itemize}
    \item Jika ada dua kotak yang memiliki jumlah permen yang sama bersebelahan, maka Josep dapat menggabung dua kotak tersebut menjadi satu kotak baru dengan banyak permen pada kotak yang baru merupakan jumlah permen-permen dari dua kotak yang digabung.
    \item Jika ada dua kotak yang memiliki jumlah permen yang sama dan dipisahkan oleh satu kotak, maka Josep dapat menggabung tiga kotak tersebut menjadi satu kotak baru. Jumlah permen pada kotak yang baru merupakan jumlah dari permen-permen di tiga kotak sebelumnya.
\end{itemize}

Josep dapat melakukan operasi tersebut sebanyak mungkin. Akan tetapi, Josep baru menyadari bahwa tantangan tersebut susah untuk ia selesaikan. Oleh karena itu, dia meminta Anda, para peserta HMIF Challenge untuk membantunya menentukan berapa permen maksimal yang dapat diberikan kepada Michella.


% \begin{enumerate}
%     \setlength\itemsep{0pt}
%     \item Contoh penomoran.
%     \item Contoh penomoran.
% \end{enumerate}

% \begin{itemize}
%     \setlength\itemsep{0pt}
%     \item Contoh membuat poin-poin.
%     \item Contoh membuat poin-poin.
% \end{itemize}

% \begin{center}
%     Teks rata tengah
%     % Contoh gambar:
%     % \includegraphics[width=300px]{image-1}
% \end{center}

\subsection*{Format Masukan}

% Baris pertama terdiri dari satu bilangan bulat positif $T$ ($1 \leq T \leq 100.000$), menyatakan banyaknya kasus uji.
% Tiap kasus uji diawali dengan bilangan $N$ ($1 \leq N \leq 100.000$).
% $N$ baris berikutnya terdiri dari 3 bilangan, dengan baris ke-$i$ menyatakan bilangan $A_i$, $B_i$, dan $C_i$.

Baris pertama berisi bilangan bulat $N$ (1 $\leq N \leq$ 400), yang menyatakan banyaknya kotak, lalu baris berikutnya berisi
$N$ buah bilangan $P_i$ (1 $\leq P_i \leq$ 1000000000) yang menyatakan banyak permen di kotak ke-i.

\subsection*{Format Keluaran}
Sebuah angka yang menyatakan jumlah permen maksimum yang bisa diberikan oleh Josep kepada Michella.
\\

\begin{multicols}{2}
\subsection*{Contoh Masukan 1}
\begin{lstlisting}
7
47 12 12 3 9 9 3
\end{lstlisting}
\columnbreak
\subsection*{Contoh Keluaran 1}
\begin{lstlisting}
48
\end{lstlisting}
\vfill
\end{multicols}


\subsection*{Penjelasan}
Josep dapat melakukan operasi dengan urutan berikut:
\begin{itemize}
    \item Menggabung dua kotak dengan 12 permen, sehingga susunannya menjadi \{47, 24, 3, 9, 9, 3\}.
    \item Menggabung dua kotak dengan 9 permen, sehingga susunannya menjadi \{47, 24, 3, 18, 3\}.
    \item Menggabung tiga kotak dengan 3, 18, dan 3 permen, sehingga susunannya menjadi \{47, 24, 24\}.
    \item Menggabung dua kotak dengan 24 permen, sehingga susunannya menjadi \{47, 48\}.
\end{itemize}
Karena kotak dengan jumlah permen 47 tidak dapat digabung dengan kotak yang berisi 48 permen, maka jumlah maksimum permen yang bisa diberikan Josep adalah 48.
% \pagebreak

\end{document}